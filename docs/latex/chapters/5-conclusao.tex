\chapter{Conclusão}

Conclui-se que os requisitos funcionais e não funcionais do sistema foram atentidos. O usuário final é capaz de realizar as principais funções de negócio, como publicar alertas e ser notificado por novos alertas nas suas proximidades. A implementação de cada nova funcionalidade do sistema ocorreu de forma intercalada entre o servidor e o aplicativo. A adoção do framework Relay~\cite{relay} gerou uma alta curva de aprendizagem inicial, ma foi compensada pelo ganhos em performance, flexibilidade e escalabilidade do aplicativo.

Em relação à validação do servidor, conclui-se que ele foi testado localmente repetidas vezes durante o processo de desenvolvimento com a ajuda do Docker~\cite{docker} para provisionamento local dos banco de dados Redis e PostgreSQL. Cada módulo do servidor foi testado com testes unitários, ou seja, testes automatizados que devem testar uma única unidade de código e que devem ignorar dependências externas. O aplicativo foi testado de forma manual e em um ambiente local.

Ainda existem, porém, dúvidas em relação à receptividade da opinião público ao projeto. Como o aplicativo não foi distribuido em lojas em razão dos custos e por se tratar, ainda, de um projeto prematuro, a ideia não foi testada com usuários reais. Considerando que o problema atacado - a segurança - é sensível a todos e a solução envolve a confiança entre pessoas anônimas, o maior desafio transcende as barreiras tecnológicas e está relacionado ao ser humano. O usuário pode, por exemplo, usar o aplicativo de forma não moderada ao agir como um ``vigilante'', ``reporter criminal'' ou até mesmo como um ``justiçeiro''. Ele pode fornecer alertas equivocados, difamatórios, racistas ou mentirosos, sejam eles mal intencionados ou não.

O uso frequente do aplicativo pode levantar questões mais filosóficas aos usuários como: ``É melhor ser totalmente informado de todos os crimes e emergências e ficar possivelmente paranoico, ou ser totalmente ignorante deles mas viver em constante perigo?''
