\chapter{Introdução}
\label{c.introducao}

% - A introdução deve incluir as respostas para as seguintes questões: 
% a) Motivação: “Por que o problema tratado é relevante?” 
% b) Solução: “O que será desenvolvido e como isto será utilizado para resolver o problema em questão?” 
% c) Desafios: “Quais serão as maiores dificuldades para alcançar a solução desejada?

% - O final da introdução deve incluir uma descrição de como o documento está estruturado (um parágrafo para indicar o conteúdo de cada seção).

Os sistemas colaborativos revolucionaram a forma como as pessoas se relacionam. Eles permitem com que uma tarefa complexa seja dividida entre pequenas tarefas simples para várias pessoas, eles rompem as barreiras físicas para a viabilizar a conexão de pessoas até então desconhecidas entre si, eles possibilitam, através de um efeito de rede, a construção de um senso de comunidade, que é autossuficiente e orgânica.

Entre esses sistemas, existem os sistemas conectadores, aqueles fornece a infraestrutura necessária para unir pessoas que são provedoras de um serviço as pessoas que são as consumidoras desse serviço. Entre os exemplos, destacam-se: Uber, AirnBnB, e iFood. Existem também os sistemas de colaboração em massa, do inglês \emph{crowdsourcing}, que se beneficia ainda mais do poder da colaboração entre milhões de pessoas, vide os grandes exemplos de sucesso: Google Docs, Wikipedia, Waze, Git.

\section{Motivação}
\label{s.motivacao}

O Brasil é um país onde as pessoas, principalmente aquelas que residem em grandes centros urbanos, vivem constantemente com um sentimento de medo de assaltos, furtos, roubos, etc. Somado à isso, muitas vezes o tempo de resposta da polícia à ocorrências é muito alto.

\section{Solução}
\label{s.solucao}

Portanto, tendo em vista a tendência de sistemas facilitadores para a chamada economia colaborativa, é apresentado neste trabalho uma solução para emponderar as pessoas a se protegerem, usando a colaboração como uma ferramenta de segurança.

Por meio de um aplicativo móvel de celular, pessoas poderiam ser alertadas e alertar outras pessoas da ocorrência de crimes, incêndios, emergências, protestos, ruas interditadas, catástrofes naturais, entre outros.

\section{Desafios}
\label{s.desafios}

Tudo que vem do usuário deve ser motivo de preocupação. É preferível que eles sejam padronizados, validados, ou até sanitizados, se for o caso. A veracidade da informação fornecida também é um risco, o usuário pode, com má intenção ou por engano, cometer injustiças sociais como calúnia, difamação, injúria racial, ou adicionar conteúdo explícito ou depreceativo. Portanto, os desafios técnicos serão grandes, porém o maior será de oferecer mecanismos para tentar mitigar esses problemas que são intrínsicos ao ser humano.
