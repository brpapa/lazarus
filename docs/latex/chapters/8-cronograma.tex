\chapter{Cronograma do projeto}
\label{c.cronograma}

% - A divisão em blocos na chapter descricao-do-projeto é importante, pois será utilizada para descrever as etapas de desenvolvimento do projeto no cronograma, assim como o acompanhamento do desenvolvimento parcial do projeto.

% - Descrever a data de início e a duração prevista para desenvolvimento de cada um dos módulos do projeto, assim como a pessoa responsável (no caso do projeto ser em equipe). Incluir as fases de documentação, desenvolvimento e testes.
% - As datas do cronograma do projeto devem ser compatíveis com as datas limites especificadas para as avaliações da disciplina de TCC.

É previsto que o desenvolvimento inicie a partir de 01/06/2021, e seja feito de forma iterativa, com a realização constante de testes unitários e de integração durante o processo de desenvolvimento.

A disponibilização do primeiro protótipo está prevista para 29/07/2021. Na seqência será coletado, portanto, os primeiros feedbacks e pontos de melhoria com os alguns usuários finais selecionados.


% TODO: SERIA BOM COLOCAR UMA FIGURA COM AS ETAPA