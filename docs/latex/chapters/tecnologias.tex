\chapter{Tecnologias pesquisadas e escolhidas}
\label{c.tecnologias}

% - Nesta seção cada equipe deverá pesquisar as alternativas de como cada um dos blocos propostos na seção anterior pode ser desenvolvido.
% - Dependendo da natureza de cada bloco, tecnologias podem ser microprocessadores, kits de desenvolvimento, linguagens de programação, protocolos de transmissão ou diferentes opções de algoritmos.
% - Após pesquisar a diferentes formas de desenvolver cada bloco o aluno deverá fazer uma escolha inicial, justificada, da tecnologia que pretende utilizar.
% - As tecnologias escolhidas devem ser descritas com mais detalhes. As tecnologias concorrentes podem ser apresentadas de maneira resumida, mas com detalhe suficiente para caracterizar sua funcionalidade, bem como seus aspectos positivos e negativos.

Para a implementaçãos, será adotada uma arquitetura baseada em microserviços e orientada à eventos. Os principais benefícios dessa escolha são: resiliência à falhas, baixo acoplamento, facilidade de escala, e facilidade para a coleta de dados.

O back-end será composto por aplicações conteinerizadas, que se comunicam de forma assíncrona via um sistema de mensageria. Aqueles serviços com maior demanda de entrada e saída (I/O-bound) serão escritos em NodeJS. E aqueles serviços com maior demanda de processamento computacional (CPU-bound) serão escritos em C\#.

Quanto à aplicação cliente, será adotado o React Native, um popular framework de código aberto mantido pelo Facebook que permite o desenvolvimento de aplicações móveis para múltiplas plataformas - iOS e Android - utilizando apenas JavaScript. 

Para suprir a infraestrutura computacional necessária, será utilizado serviços oferecidos por provedores de computação em nuvem, especialmente a Amazon Web Services (AWS). Entre as principais vantagens estão: alta disponibilidade de serviços, pagamento sob demanda, e agilidade no desenvolvimento.