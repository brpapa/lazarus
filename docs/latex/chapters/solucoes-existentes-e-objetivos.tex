\chapter{Soluções existentes e objetivos da proposta}
\label{c.solucoes-existentes-e-objetivos}

\section{Soluções existentes}
\label{c.solucoes-existentes}

% - Deve-se fazer uma pesquisa de soluções existentes para o mesmo problema que deseja resolver.
% - Soluções existentes podem ser projetos comerciais ou acadêmicos.
% - As soluções pesquisadas devem ser descritas com profundidade suficiente para que fique caracterizada sua competência em resolver o problema em questão, bem como seus aspectos positivos e negativos.
% - Na ausência de soluções, devem ser incluídas soluções que tratem partes da solução proposta ou problemas similares.
% - Ao final da seção deverá ser feita uma análise crítica das soluções existentes.
% - Como base nessa análise, os objetivos do projeto proposto deverão ser descritos de forma a caracterizar claramente suas diferenças em relação aos projetos existentes.

\subsection{Mundo}
\label{ss.mundo}

O aplicativo \emph{Citizen}, desenvolvido pela \emph{sp0n, Inc.}, foi lançado originalmente em 2016 com o nome de \emph{Vigilante} em alguns grandes centros dos Estados Unidos. Em junho de 2020, ele possuía cerca de 5 milhões de usuários ativos. Seu sucesso vem da sua robustez, usabilidade, velocidade e praticidade. Entre as principais funcionalidades destacam-se o envio de alertas de segurança baseado na localização em tempo real, o acompanhamento pelos usuários dos alertas que estão em andamento, a transmissão de vídeos ao vivo e a possibilidade de adicionar comentários. As suas notificações já ajudaram pessoas à evacuarem de prédios em chamas e ônibus escolares à escaparem de ataques terroristas. 

A empresa dona do aplicativo, \emph{sp0n, Inc}, possui antenas de rádio nas cidades suportadas para que as chamadas telefônicas da polícia local sejam monitadoras, permitindo com que operadores especializados da empresa as filtrem e publiquem alertas no aplicativo. Em razão disso, o aplicativo \emph{Citizen} tem sua atuação dependente dos orgãos públicos. No presente momento, Agosto de 2020, ele atua em apenas 60 cidades dos Estados Unidos.

% Cidades suportadas: Atlanta, Austin, Baltimore, Boston, Cleveland, Cincinnati, Columbus, Charlotte, Chicago, Detroit, Miami-Date, Houston, Indianapolis, Los Angeles, Louisville, Minneapolis, New York City, Newark, Portland, Philadelphia, Phoenix, San Franciso, San Diego, Stockton, Tucson, Toledo, Washington

\subsection{Brasil}
\label{ss.brasil}

% https://www.techtudo.com.br/tudo-sobre/bo-coletivo.html

No Brasil, o aplicativo \emph{BO coletivo} permite com que os usuários colaborem com um mapeamento coletivo de assaltos, furtos, roubos e sequestros. Porém, ele é útil apenas para consultas posteriores, não oferecendo mecanismos para ações preventivadas em tempo real, e também, em relação à usabilidade, o aplicativo não permite a navegação pelo mapa ao digitar uma região desejada.

% https://33giga.com.br/app-policia-popular-transforma-usuarios-em-agentes-contra-o-crime/
% https://capozzielli.gdigital.com.br/usuario/

O aplicativo \emph{SP+ segura}, por outro lado, permite com que pessoas informem e sejam informadas de alertas em tempo real sobre episódios de risco em que pessoas se encontram ou presenciam. Porém, ele não oferece interação de chat entre usuários para eventuais discussões, não suporta o envio de vídeos, e não possui nenhum mecanismo de rankeamento dos alertas.

\subsection{Análise}
\label{ss.analise}

Dada as soluções apresentadas, o \emph{Citizen} é a maior referência do segmento no mundo. Porém, apesar de seu enorme sucesso, ele é restrito à apenas algumas cidades dos Estados Unidos, e não apresenta previsão de expansão para o Brasil, país que possui algumas soluções para o tema, porém são mais limitadas. Tendo isso em vista, surgiu-se a oportunidade de oferecer uma solução alternativa que seja melhor.

\section{Objetivos}
\label{s.objetivos}

% Esta especificação tem por objetivo definir em detalhes o trabalho que será desenvolvido na disciplina de TCC. Nessa fase, o aluno deve cumprir quatro objetivos fundamentais:

% a) Descrever claramente o problema que deseja resolver e sua motivação.  
% b) Pesquisar soluções existentes para o problema que deseja resolver.
% c) Propor uma solução alternativa para o problema, caracterizando as diferenças com as soluções existentes.
% d) Pesquisar as tecnologias e ferramentas que poderão ser utilizadas em seu projeto, a fim de avaliar sua viabilidade.

Este trabalho tem como objetivo desenvolver uma aplicação móvel onde as pessoas possam se manter conscientes, em tempo real, de situações perigosas que estão ocorrendo nas suas proximidades, como crimes, incêndios, ameaças, protestos, ruas interditadas, catástrofes naturais. O aplicativo deve oferecer a infraestrutura para o desenvolvimento de um senso de comunidade. A integração dos alertas com os chamados das polícias locais não está incluso no escopo do projeto.
