\chapter{Conclusão}
\label{c.conclusao}

% - A conclusão deve destacar os principais aspectos do projeto, os ganhos esperados, os principais desafios a serem enfrentados.
% - Deve-se dar uma avaliação geral da viabilidade do projeto.

Conclui-se que os requisitos funcionais e não funcionais do sistema foram atentidos. O usuário final é capaz de realizar as principais funções de negócio, como publicar alertas e ouvir por alertas nas proximidades.

Dentre os próximos passos, existem, porém, desafios em relação à distribuição e à recepção da opinião pública ao projeto. 

O aplicativo não foi publicado em lojas em virtude dos custos e por se tratar, ainda, de um projeto prematuro.

O projeto também não foi testado com usuários reais, e portanto não se sabe como eles receberiam a ideia. O projeto visa atacar um problema sensível a todos: a segurança. O desafio, portanto, transcende as barreiras tecnológicas e tem suas raízes em questões culturais. As questões mais críticas são aquelas relacionadas ao ser humano. Mesmo com uma plataforma excepcional, e que permita a construção de um senso de comunidade, os valores das pessoas que o alimentam sempre estará em primeiro plano. Alguns países com uma cultura de valorização do coletivismo podem usufruir melhor da plataforma.

