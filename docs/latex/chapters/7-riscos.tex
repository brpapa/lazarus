\chapter{Riscos}
\label{c.riscos}

% - Nesta seção as equipes devem fazer uma análise dos problemas potenciais do projeto e do impacto desses problemas no seu sucesso ou fracasso (seguir a metodologia de engenharia de software).
% - Riscos típicos são:
%   - Aqueles relacionados à falta de componentes no mercado; 
%   - Atrasos na entrega de materiais;
%   - Falta de recursos materiais;
%   - Erro de estimativa da complexidade dos módulos;
%   - Escolha incorreta de implementação de algum módulo do sistema devido a tecnologia ser pouco conhecida.
% - Para os riscos que tiverem alto impacto na viabilidade do projeto, deverá ser previsto um plano de contingência.
% - Geralmente na forma de alteração da tecnologia de desenvolvimento de um módulo ou alteração de uma funcionalidade prevista.

O fato do sistema ser totalmente alimentado por usuários, e sobre um assunto sensível, faz com os pontos de atenção sejam muitos. O usuário pode, por exemplo, usar o aplicativo de forma não moderada, agindo como um "vigilante", "reporter criminal" ou até mesmo um "justiçeiro". 

O usuário pode não ser confiável. Ele pode fornecer alertas equivocados, difamatórios, racistas ou mentirosos, seja com intenção ou não. Nesses casos, as pessoas que tiverem sua imagem revelada, podem acabar recorrendo à justiça.

O usuário que mora em regiões mais perigosas pode ser alimentado constantemente por um sentimento de medo.

Questões de privacidade devem ser claras. O usuário precisa estar ciente que está compartilhando a sua localização o tempo todo, inclusive enquanto não está com o aplicativo aberto.

Questões filóficas também podem surgir entre os pensamentos dos usuários. Será que é melhor ser totalmente informado de todos os crimes e ficar possivelmente paranoico, ou ser totalmente ignorante deles mas viver em constante perigo?
