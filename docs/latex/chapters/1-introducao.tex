\chapter{Introdução}
\label{c.introducao}

Os sistemas colaborativos revolucionaram a forma como as pessoas se relacionam. Eles permitem com que uma tarefa complexa seja dividida em pequenas tarefas simples e distribuídas entre várias pessoas. Eles rompem as barreiras físicas para a viabilizar a conexão de pessoas até então desconhecidas entre si. E eles possibilitam, através de um efeito de rede, a construção de um senso de comunidade autossuficiente e orgânico.

Dentre esses sistemas, existem os sistemas conectadores, que são plataformas que fornecem toda a infraestrutura necessária para unir pessoas que são provedoras de um serviço às pessoas que são as consumidoras desse serviço, como aplicativos de transporte que permitem a busca por motoristas baseada na localização. Existem também os sistemas de colaboração em massa, do inglês \emph{crowdsourcing}, que se beneficia ainda mais do poder da colaboração entre milhões de pessoas, como, por exemplo, o Wikipedia ~\cite{wikipedia}, a maior enciclopédia livre da internet.

A tecnologia, aliada à colaboração entre usuários, pode também ser uma importante ferramenta de segurança. Um grande exemplo de sucesso é o aplicativo Waze ~\cite{waze}, onde o usuário pode contribuir com alertas de acidentes de trânsito e com informações de rotas que possam estar congestionadas. Há, portanto, uma relação ganha-ganha para todas as partes envolvidas.

% \section{Motivação}

A segurança pública no Brasil é um problema relevante. O Brasil é um país onde as pessoas, sobretudo aquelas que residem em grandes centros urbanos, vivem constantemente com uma sensação de insegurança em decorrência do medo de assaltos, furtos, roubos e crimes em geral que, apesar de ser dever do Estado garantir a segurança pública, muitas vezes o tempo de resposta é muito alto. Somado à isso, catástrofes naturais são cada vez mais comuns em virtude do aquecimento global.

% \section{Solução}

Para resolver o problema em questão, e tendo em vista a tendência do uso de sistemas facilitadores na economia colaborativa, é apresentado neste trabalho uma solução para empoderar as comunidades locais a se protegerem e proteger uns aos outros usando a tecnologia como uma ferramenta de segurança. Por meio de um aplicativo móvel de celular, pessoas podem ser alertadas e alertar outras pessoas próximas sobre a ocorrência de crimes, tiroteios, incêndios, emergências, protestos, ruas interditadas, catástrofes naturais, entre outros incidentes que coloquem em risco a segurança das pessoas.

% \section{Desafios}

Para a implementação da solução desejada houve inúmeros desafios. Entre os desafios técnicos, destacam-se o sistema de notificações e o mecanismo de rastreamento em tempo real da localização do usuário. Porém, o maior desafio foi o de tentar mitigar problemas intrínsicos ao ser humano, uma vez que a veracidade da informação fornecida pelos usuários é um risco. O usuário pode, com má intenção ou por engano, cometer injustiças sociais como calúnia, difamação, injúria racial, ou adicionar conteúdo explícito ou depreceativo, por exemplo.

Além da introdução, este documento está estruturado em mais quatro capítulos. O Capítulo~\ref{c.solucoes-existentes-e-objetivos} descreverá quais são as principais soluções existentes para o mesmo problema no Brasil e no mundo, elencando os seus aspectos positivos e negativos, para que uma análise crítica das mesmas seja feita e, com base nessa análise, os objetivos deste trabalho serão descritos. O Capítulo~\ref{c.detalhamento-do-problema} abordará a natureza do problema e definirá os requisitos funcionais e não funcionais da solução. O Capítulo~\ref{c.descricao-do-sistema} descreverá em detalhe o sistema desenvolvido, as decisões de arquitetura e as principais tecnologias utilizadas. Por fim, no Capítulo~\ref{c.conclusao} será apresentado os resultados da solução desenvolvida e como ele foi testado durante todo processo de desenvolvimento.
