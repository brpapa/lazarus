\chapter{Introdução}

Os sistemas colaborativos revolucionaram a forma como as pessoas se relacionam. Eles permitem com que uma tarefa complexa seja dividida em pequenas tarefas simples e distribuídas entre várias pessoas. Eles rompem as barreiras físicas para a viabilizar a conexão de pessoas até então desconhecidas entre si. E eles possibilitam, através de um efeito de rede, a construção de um senso de comunidade autossuficiente e orgânico.

Dentre esses sistemas, existem os sistemas conectadores, que são plataformas que fornecem toda a infraestrutura necessária para unir pessoas que são provedoras de um serviço às pessoas que são as consumidoras desse serviço, como aplicativos de transporte que permitem a busca por motoristas baseada na localização. Existem também os sistemas de colaboração em massa, do inglês \emph{crowdsourcing}, que se beneficia ainda mais do poder da colaboração entre milhões de pessoas, como, por exemplo, o Wikipedia ~\cite{wikipedia}, a maior enciclopédia livre da internet.

A tecnologia, aliada à colaboração entre usuários, pode também ser uma importante ferramenta de segurança. Um grande exemplo de sucesso é o aplicativo Waze ~\cite{waze}, onde o usuário pode contribuir com alertas de acidentes de trânsito e com informações de rotas que possam estar congestionadas. Há, portanto, uma relação ganha-ganha para todas as partes envolvidas.

\section{Motivação}
% “Por que o problema tratado é relevante?” 

O Brasil é um país onde as pessoas, sobretudo aquelas que residem em grandes centros urbanos, vivem constantemente com uma sensação de insegurança em decorrência do medo de assaltos, furtos, roubos e crimes em geral que, apesar de ser dever do Estado garantir a segurança pública, muitas vezes o tempo de resposta é muito alto. Somado à isso, catástrofes naturais são cada vez mais comuns em virtude do aquecimento global.

\section{Solução}
% “O que será desenvolvido e como isto será utilizado para resolver o problema em questão?” 

Tendo em vista a tendência do uso de sistemas facilitadores para a economia colaborativa, é apresentado neste trabalho uma solução para emponderar as comunidades locais a se protegerem e proteger uns aos outros usando a tecnologia como uma ferramenta de segurança. Por meio de um aplicativo móvel de celular, pessoas poderiam ser alertadas e alertar outras pessoas próximas sobre a ocorrência de crimes, tiroteios, incêndios, emergências, protestos, ruas interditadas, catástrofes naturais, entre outros incidentes que coloquem em risco a segurança das pessoas.

\section{Desafios}
% “Quais serão as maiores dificuldades para alcançar a solução desejada?

A implementação da solução possui seus desafios técnicos, como o sistema de notificações e o mecanismo de rastreamento em tempo real da localização do usuário. Porém, a própria solução também trás consigo grantes desafios. Qualquer dado que vem do usuário deve ser motivo de preocupação e, portanto, é preferível que eles sejam padronizados, validados, ou até sanitizados, se for o caso. A veracidade da informação fornecida por eles também é um risco, o usuário pode, com má intenção ou por engano, cometer injustiças sociais como calúnia, difamação, injúria racial, ou adicionar conteúdo explícito ou depreceativo. Portanto, o maior desafio é o de oferecer formas para tentar mitigar esses problemas intrínsicos ao ser humano.

% \section{Estrutura do documento}
% TODO: O final da introdução deve incluir uma descrição de como o documento está estruturado (um parágrafo para indicar o conteúdo de cada seção).