\chapter{Detalhamento do problema}
\label{c.detalhamento-do-problema}

% - Nesta seção o problema e sua motivação devem ser apresentados em detalhes.
% - Devem ser caracterizados: 
% - a) A natureza do problema. 
% - b) A importância do problema: quais segmentos da sociedade são afetados por ele. 
% - c) O impacto econômico que o problema causa atualmente e os benefícios esperados com sua solução. 
% - d) Os requisitos, funcionais, de desempenho e de custo, que uma possível solução para o problema deverá satisfazer.

\section{Natureza do problema}
\label{s.natureza-do-problema}

% TODO: citar fontes aqui do pq seguranca publica no brasil é problema
% TODO: trazer dados?
A segurança pública no Brasil é dever do Estado, porém o serviço oferecido não é eficiente. O Estado está distante do cotidiano do cidadão, o tempo de resposta dos chamados policiais muitas vezes não é rápido o bastante, entre outros.

Sendo assim, um ambiente online que ofereça a infraestrutura necessária para a construção de verdadeiras comunidades locais, onde a confiança é estabelecidade com o tempo, com a ajuda de vídeos e images dos alertas que forem reportados, pode ajudar a amenizar o problema.

\section{Requisitos}
\label{s.requisitos}

Dado a natureza do problema, uma solução para o problema deve atender os seguintes requisitos.

\subsection{Requisitos funcionais}
\label{s.requisitos-funcionais}

\begin{alineas}
	\item Usuários devem ser capazes de publicar alertas, que devem suportar o upload de fotos e vídeos de curta duração;
 	\item Usuários visualizam um mapa com os alertas mais recentes nas suas proximidades;
	\item Usuários devem ser notificados quando alertas forem publicados nas suas proximidades.
\end{alineas}

\subsection{Requisitos não funcionais}
\label{s.requisitos-nao-funcionais}

\begin{alineas}
	\item o sistema deve ser seguro;
	% \item o sistema deve ser altamente disponível, e em virtude disso, é aceitável uma temporária inconsistência;
	\item o sistema deve ser altamente confiável, ou seja, qualquer foto ou vídeo carregado nunca deve ser perdido;
	\item os usuários do aplicativo devem ter uma experiência de uso em tempo real;
	\item o sistema deve suportar uma alta carga de leituras com latência mínima, sendo tolerável uma latência maior em escritas.
\end{alineas}
