\chapter{Procedimentos de validação do projeto}
\label{c.validacao}

% - Nesta seção devem ser indicados como os diversos módulos do projeto serão testados e validados individualmente e em conjunto.
% - Os testes gerais, que incluem o funcionamento do sistema do “ponto de vista do usuário”
% - Os testes do ambiente são usualmente denominados de “caixa preta”.
% - Os testes que permitem validar os módulos do projeto individualmente são usualmente denominados de “caixa branca”.
% - Nesta fase, a descrição dos testes poderá ser bem resumida, indicando principalmente os critérios de aceite do sistema e de cada um de seus módulos em termos do cumprimento de seus aspectos funcionais e de desempenho.

Cada módulo será testado de forma isolada, tendo em vista que eles estarão desacoplados por um sistema de mensageria.

Um entregável funcional chegará nas mãos do usuário final apenas após primeira entrega. Após o primeiro MVP, será coletado pontos de melhoria com alguns usuários selecionados. E após a finalização do projeto, a aplicação será monitadora, e os usuários serão questionados com pesquisas de satisfação.